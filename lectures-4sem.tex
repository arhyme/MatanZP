\documentclass[a5paper]{article}
\usepackage[cm]{fullpage}
\usepackage{cmap}
\usepackage[T2A]{fontenc}
\usepackage[utf8]{inputenc}
\usepackage[english,russian]{babel}
\usepackage{amsmath,amssymb,amsthm}
\usepackage{hyperref}
\usepackage{enumitem} 
\usepackage{euscript}
\usepackage{icomma}

\newlist{enum}{enumerate}{10}
\setlist[enum]{label*=\arabic*.}
\usepackage{tikz}

\newlist{sectenum}{enumerate}{10}
\setlist[sectenum]{
  label=\arabic{section}.\arabic{*}
}

\newcommand*{\hm}[1]{#1\nobreak\discretionary{}
	{\hbox{$\mathsurround=0pt #1$}}{}}

\newcounter{through}

\theoremstyle{plain}
\newtheorem{theorem}[through]{Теорема}
\newtheorem{corollary}[through]{Следствие}
\newtheorem{lemma}[through]{Лемма}
\newtheorem{task}[through]{Задача}

\theoremstyle{definition}
\newtheorem{definition}[through]{Определение}
\newtheorem{example}[through]{Пример}

\numberwithin{through}{section}
\numberwithin{equation}{section}

\DeclareMathOperator{\supp}{supp}
\DeclareMathOperator{\rank}{rank}

\hypersetup{
  pdftitle={lecture-4sem},
  pdfauthor={},
  colorlinks=true,
  linkcolor=blue
}

\begin{document}
	
	\selectlanguage{russian}
	
	\title{Лекции \\
	по математическому анализу:\\
	многообразия, дифференциальные формы}
	\maketitle
	
	\begin{abstract}
		Записки создавались студентами Механико-математического факультета НГУ 
		с лекций, прочитанных С. Г. Басалаевым.
	\end{abstract}

\clearpage
\tableofcontents

\clearpage


\section{Многообразия}

\subsection{Многообразия без края}

\begin{definition}
	\label{ManifoldNoBoundary}
	Множество $M \in \mathbb{R}^n$ называется $C^r$-гладким $k$-мерным
	многообразием  без края, если для каждого $x_0 \in M$ существует $U$ - окрестность 
	$x_0$ и существует $ C^r$-диффеоморфизм $\Phi : U \to \Phi (U)$, такой что 
	$\Phi(x_0)=0$ и $\Phi(U \cap M) = V \times \{0\}^{n-k}$, где $V$ - окрестность нуля в $R^n$.
\end{definition}

При $r = 0$ многообразие называется топологическим, при $r > 0$ многообразие называется дифференцируемым.


Виды многообразий:
\begin{enumerate}	
	\item
	Набор изолированных точек $(k = 0)$.
	
	\begin{center}
		\begin{tikzpicture}
		\filldraw [black] (0.89,0.12) circle (2pt);
		\filldraw [black] (0.48,0.76) circle (2pt);
		\filldraw [black] (0.48,0.26) circle (2pt);
		\filldraw [black] (1,0.6) circle (2pt);
		\filldraw [black] (0, 0.5) circle (2pt);
		\draw [red] (0,0.5) circle (6pt);
		\end{tikzpicture}	
	\end{center}
	
	
	\item
	Набор кривых, в том числе с выколотыми концами, а также замкнутые ($k = 1$).
	\begin{center}
		\begin{tikzpicture}
		\draw (0,0) .. controls (1,-1) and (2,-1) .. (3,-1);
		\filldraw [white] (0,0) circle (3pt)
		(3,-1) circle (3pt);
		\draw [gray] (0,0) circle (3pt)
		(3,-1) circle (3pt);
		\end{tikzpicture}
	\end{center}
	\item
	Поверхности ($k = 2$).
	
	\begin{center}
		\begin{tikzpicture}
		\draw (0,0) circle (1.5cm);
		\draw (-1.5,0) arc (180:360:1.5 and 0.6);
		\draw[dashed] (1.5,0) arc (0:180:1.5 and 0.6);
		\fill[fill=black] (0,0) circle (1pt);
		\end{tikzpicture}
	\end{center}
	
\end{enumerate}

\begin{example}
	\
	\begin{itemize}
		\item 
		Пара параллельных прямых -- многообразие,
		\item 
		Пара непересекающихся плоскостей -- многообразие,
		\item 
		Плоскость и прямая -- не многообразие, так как их размерности не совпадают,
		\item 
		Пара пересекающихся прямых с выколотой точкой пересечения -- многообразие.
		\begin{center}
			\begin{tikzpicture}
			\draw (0,0) -- (4,2);
			\draw (4,0) -- (0,2);
			\filldraw (2,1) [white] circle (4pt);
			\draw (2,1) [red] circle (4pt);
			\end{tikzpicture}
		\end{center}
		
		
	\end{itemize}
	
\end{example}

Теперь рассмотрим способы задания $k-$мерных многообразий.

\begin{theorem}%[Об открытом множестве]% 
	\label{OpenSetManifold}
	Пусть $U \subseteq \mathbb{R}^n$ -- окрытое множество, тогда $U$ является 
	$n$-мерным многообразием.
\end{theorem}

Первый способ 
\begin{proof}
	Напомним, что множество называется \textit{открытым}, если для любая его точка $x_0$ лежит в некоторой окрестности.\\
	Отображение $\Phi$ можно определить следующим образом:
	$\Phi(x)=x-x_0$ -- сдвиг в ноль, переводит окрестность $x_0$ в окрестность нуля.
\end{proof}

\begin{theorem}[О графике]
	Пусть $U \subset \mathbb{R}^n$ -- открытое множество, $f: U \to \mathbb{R}, f \in C^r$, тогда график этой функции $\Gamma_f = 
	\{ (x, f(x))  \mid x \in U \}$ - $C^r$-гладкое $n$-мерное многообразие в $R^{n+1}$.
\end{theorem}

\begin{proof}
	Определим отображение $\Phi : U \times \mathbb{R} \to U \times \mathbb{R}$ следующим образом $\Phi(x, y) = (x, y - f(x))$, тогда:
	
	$\Phi(x_1, \ldots, x_n, y) = 
	\begin{pmatrix}
	x_1 \\
	\vdots \\
	x_n \\
	y - f(x_1, \ldots, x_n)
	\end{pmatrix} \quad
	\left| D \Phi \right|=
	\begin{vmatrix}
	E& 0 \\
	-\frac{\partial f}{\partial x_i} & 1
	\end{vmatrix}
	= 1
	$
	
	Отображение $\Phi$ является $C^r$-диффеоморфизмом и $\Phi(x,f(x)) = (x, 0)$, т.е. $\Phi(\Gamma_f) = U \times \{0\}$
\end{proof}

\begin{theorem}[О локальном вложении]
	Пусть $U \subset \mathbb{R}^k$ - открытое множество, $f: U \to \mathbb{R}^n, f \in C^r$ и $k \leq n $. Тогда, если $t^0 \in U и \rank Df(t^0)) = k$, то существует $U$-окрестность $t^0$, такая что $f(U)$ является $C^r$-гладким $k$-мерным многообразием.
\end{theorem}

\begin{proof}
	Так как $\rank Df(t^0) = k$, следовательно набор векторов $\{ \frac{\partial f}{ \partial x_1}, \ldots, \frac{\partial f}{\partial x_k}\}$  является линейно независимым, т.е. его можно дополнить до базиса $\mathbb{R}^n$.
	
	Пусть $\{ \frac{\partial f}{ \partial x_1}, \ldots, \frac{\partial f}{\partial x_k}\, v_{k+1}, \ldots, v_n\}$ - базис в $\mathbb{R}^n$.
	
	Определим $\Phi : U \times \mathbb{R}^{n-k} \to \mathbb{R}^n$ так, что $\Phi(t_1, \ldots, t_k, s_{k+1}, \ldots, s_n)=f(t_1, \ldots, t_k) +s_{k+1}v_{k+1} + \ldots + s_n v_n$.
	
	$D \Phi = [ \frac{\partial f}{ \partial x_1}, \ldots, \frac{\partial f}{\partial x_k}\, v_{k+1}, \ldots, v_n]$, следовательно $\det D \Phi \not= 0$.
	
	По теореме об обратной функции существует $W$ - окрестность $(t_0, 0)$, такая что $\Phi : W \to \Phi(W) - C^r$-диффеоморфизм. Выберем $V \times (-h, h) \in W$, так что $V$ - окрестность $x_0$ в $\mathbb{R}^k$, тогда $\Phi^{-1}(f(V))=V\times{0}^{n-k}$.
	
\end{proof}

\begin{definition}
	Пусть $x^0$ - решение системы уравнений:
	
	\begin{equation*}
	\begin{cases}
	f_1(x_1, \ldots, x_n) = a_1
	\\
	\ldots
	\\
	f_k(x_1, \ldots, x_n) = a_k
	\end{cases}
	\end{equation*}
	тогда $x^0$ называется регулярным, если 
	$\rank \, \frac{\partial f_i}{\partial x_j} (x^0) = k$.
\end{definition}

\begin{theorem}[О решении системы уравнений]
	Пусть $U \subset \mathbb{R}^k$ - открытое множество, $f_1, \ldots, f_k : 
	U \to \mathbb{R}$ и $f_i \in C^r \; \forall \, i \leq k$, тогда множество регулярных решений системы уравнений
	\begin{equation*}
	\begin{cases}
	f_1(x_1, \ldots, x_n) = a_1
	\\
	\ldots
	\\
	f_k(x_1, \ldots, x_n) = a_k
	\end{cases}
	\end{equation*}
	представляет собой $C^r$-гладкое $(n-k)$-мерное многообразие.
\end{theorem}
\begin{proof}	
	Пусть $x^0$ - регулярное решение, тогда:
	
	\begin{equation*}
	D f(x^0) = \begin{pmatrix}
	\frac{\partial f_1}{ \partial x_1}& \ldots& \frac{\partial f_1}{ \partial x_n} \\
	\vdots& \quad& \vdots \\
	\frac{\partial f_k}{ \partial x_1}& \ldots& \frac{\partial f_k}{ \partial x_n}
	\end{pmatrix} (x^0)
	\end{equation*}
	
	Матрица состоит из $n$ столбцов, где $k$ из них линейно независимы.
	
	%$f(x,y) = 0$, $\det \frac{\partial f}{ \partial y} \not= 0 \implies в некоторой окрестности y=g(x)$.%
	
	
	В некоторой окрестности $x^0$:	
	
	$x_{n-k+1} = g_{n-k+1}(x_1, \ldots, x_{n-k})$
	
	$\ldots$
	
	$x+n = g_n(x_1, \ldots, x_{n-k})$
	
	Множество решений по теореме о локальном вложении является многообразием:
	
	$x_1=x_1$
	
	$\ldots$
	
	$x_k=x_k$
	
	$x_{n-k+1} = g_{n-k+1}(x_1, \ldots, x_{n-k})$
	
	$\ldots$
	
	$x+n = g_n(x_1, \ldots, x_{n-k})$
	
\end{proof}

\subsection{Многообразия с краем}

\begin{definition}
	$\mathbb{R}_+^k = \{(x_1, \ldots, x_k) : x_k > 0\}$ - верхнее полупространство ($\mathbb{R}_-^k = \{(x_1, \ldots, x_k) : x_k < 0\}$ - нижнее полупространство).
\end{definition}

\begin{definition}
	Пусть $V \subseteq R^k$ - окрестность нуля, тогда $V \cap \mathbb{R}_+^k$ называется полуокрестностью нуля. В ее основании лежит $(k-1)$-мерная окрестность нуля.
\end{definition}

\begin{definition}
	\label{ManifoldWithBoundary}
	
	Множество $M \in \mathbb{R}^n$ называется $C^r$-гладким $k$-мерным
	многообразием с краем, если для каждого $x_0 \in M$ существует $U$ - окрестность $x_0$ и существует $ C^r$-диффеоморфизм $\Phi : U \to \Phi (U)$, такой что $\Phi(x_0)=0$ и, либо $\Phi(U \cap M) = V \times \{0\}^{n-k}$ (тогда $x_0$ - внутреняя точка), либо 
	$\Phi(U \cap M) = (V \cap \mathbb{R}_+^k ) \times \{0\}^{n-k}$ (тогда 
	$x_0$ - крайняя точка).
	
\end{definition}


Примеры многообразий с краем:

\begin{enumerate}	
	\item
	Края нет при $k = 0$
	
	\item
	Край незамкнутой кривой $(k = 1)$ - это ее концевые точки. У замкнутых и неограниченных кривых края нет.
	\begin{center}
		\begin{tikzpicture}
		\draw (0,0) .. controls (1,-1) and (2,-1) .. (3,-1);
		\filldraw [blue] (0,0) circle (2pt)
		(3,-1) circle (2pt)
		(1.8,-0.925) circle (2pt);
		\draw [red] (1.8,-0.925) circle (5pt);
		
		\end{tikzpicture}
	\end{center}
	\item
	При $k = 2$, внутренними точками поверхности являются те, которые лежат внутри нее вместе со своей некоторой окресности, остальные являются краевыми.
	
	
	
\end{enumerate}

\begin{definition}
	Пусть $M$- многообразие, тогда $\partial M = \{x \mid x$ - краевая точка $M\}$ - множество краевых точек многообразия $M$.
\end{definition}

\begin{theorem}[О крае]
	Пусть $M$ - $C^r$-гладкое $k$-мерное многообразие с краем, тогда множество его краевых точек $\partial M$ является $C^r$-гладким $k$-мерным многообразием без края ($\partial \partial M = \emptyset$).
\end{theorem}

\begin{proof}
	Пусть $x \in \Phi(U \cap M)$, тогда $x=(x_1, \ldots, x_{k-1}, x_k, 0, \ldots, 0)$. %Если $x_k > 0$, то $x$ - внутреняя точка многообразия.%
	Если $x_k = 0$, то $x \in \partial M$ - краевая точка. Отображение $\Phi$ переводит все краевые точки $U \cap M$ в $(k-1)$-мерную плоскость (основание полупространства). Основание - $(k-1)$-мерная окрестность нуля, следовательно $\Phi(U \cap M) = W \times \{0\} \times \{0\}^{n-k}$ - выполнено определение многообразия.
\end{proof}

\begin{theorem}[О решении системы уравнений и неравенства]
	Пусть $U \subset \mathbb{R}^k$ - открытое множество, $f_1, \ldots, f_{k+1} : 
	U \to \mathbb{R}$, $f_i \in C^r \; \forall \, i \leq k$ и $\rank \, \frac{\partial f_i}{\partial x_j} (x^0) = k + 1$,  тогда множество регулярных решений системы уравнений
	\begin{equation*}
	\begin{cases}
	f_1(x_1, \ldots, x_n) = a_1
	\\
	\ldots
	\\
	f_k(x_1, \ldots, x_n) = a_k
	\\
	f_{k+1}(x_1, \ldots, x_n) \geq a_{k+1}
	\end{cases}
	\end{equation*}
	представляет собой $C^r$-гладкое $(n-k)$-мерное многообразием с краем (внутрение точки - решение строгого неравенства, край - решение $(k+1)$ уравнений).
\end{theorem}
\begin{proof}
	Решение неравенства $f_{k+1}(x_1, \ldots, x_n) > a_{k+1}$ является открытым множеством, и по теореме \ref{OpenSetManifold} задает многообразие.
	
	Решение системы уравнений 
	\begin{equation*}
	\begin{cases}
	f_1(x_1, \ldots, x_n) = a_1
	\\
	\ldots
	\\
	f_k(x_1, \ldots, x_n) = a_k
	\\
	f_{k+1}(x_1, \ldots, x_n) = a_{k+1}
	\end{cases}
	\end{equation*}
	
	задает $(k-1)$ мерную поверхность - край многообразия.
\end{proof}

\begin{definition}
	Множество $M \in \mathbb{R}^k$ называется $k$-мерным кусочно-гладким многообразием, если:
	\begin{enumerate}
		\item $M$ -- $k$-мерное топологическое многообразие.
		\item Существует разбиение $M = \tilde{M} \cup (\bigcup_{i=0}^{n} Z_i) $, такое что $\tilde{M}$ - гладкое $k$-мерное многообразие, а $Z_i$ - кусочно гладкие многообразия размерности $l \leq k - 1$.
	\end{enumerate}
\end{definition}

\begin{example}
	Куб.
	\begin{center}
		
		\begin{tikzpicture}
			\coordinate (O) at (0,0,0);
			\coordinate (A) at (0,2,0);
			\coordinate (B) at (0,2,2);
			\coordinate (C) at (0,0,2);
			\coordinate (D) at (2,0,0);
			\coordinate (E) at (2,2,0);
			\coordinate (F) at (2,2,2);
			\coordinate (G) at (2,0,2);
			
			\draw[blue,fill=orange!80] (O) -- (C) -- (G) -- (D) -- cycle;% Bottom Face
			\draw[blue,fill=blue!30] (O) -- (A) -- (E) -- (D) -- cycle;% Back Face
			\draw[blue,fill=red!10] (O) -- (A) -- (B) -- (C) -- cycle;
			\draw[blue,fill=red!20,opacity=0.8] (D) -- (E) -- (F) -- (G) -- cycle;
			\draw[blue,fill=red!20,opacity=0.6] (C) -- (B) -- (F) -- (G) -- cycle;
			\draw[blue,fill=red!20,opacity=0.8] (A) -- (B) -- (F) -- (E) -- cycle;
			
			\coordinate [label=left:\textcolor{blue}{\text{вершины}}]
			(V) at (-1.5,-1);
			\draw[->] (V) -- (C) ;
			\draw[->] (V) -- (B);
			\draw[->] (V) -- (A);
			
			\coordinate [label=left:\textcolor{blue}{\text{ребра}}]
			(R) at (5.1,2);
			\draw[->] (4,2) -- (2,2,1);
			\draw[->] (4,2) -- (2,1,0);
			\draw[->] (4,2) -- (2,0,1);
			
			\coordinate [label=left:\textcolor{blue}{\text{ребра}}]
			(H) at (5.1,2);

		\end{tikzpicture}
	\end{center}
	
\end{example}

\begin{theorem}
	Пусть $U \subset \mathbb{R}^k$ - открытое множество, $f_1, \ldots, f_{k}, \ldots ,f_{k+l} : 
	U \to \mathbb{R}$, $f_i \in C^r \; \forall \, i \leq k + l$ и $\rank \, \frac{\partial f_i}{\partial x_j} (x^0) = k + l$,  тогда множество регулярных решений системы уравнений
	\begin{equation*}
	\begin{cases}
	f_1(x_1, \ldots, x_n) = a_1
	\\
	\ldots
	\\
	f_k(x_1, \ldots, x_n) = a_k
	\\
	f_{k+1}(x_1, \ldots, x_n) \geq a_{k+1}
	\\
	\ldots
	\\
	f_{k+l}(x_1, \ldots, x_n) \geq a_{k+l}
	\end{cases}
	\end{equation*}
	представляет собой $C^r$-гладкое $(n-k)$-мерное многообразием с краем.
\end{theorem}

\begin{example}
	в $\mathbb{R}^3$. $x^2+y^2\leq 1, x\geq 0, z\geq 0, z\leq 1$
	
	\begin{center}
		
		\begin{tikzpicture}[->]
		\draw (0,0) -- (xyz cs:x=4);
		\draw (0,0) -- (xyz cs:y=4.5);
		\draw (0,0) -- (0.52,-1.06,1.6);
		\draw (1.25,0) arc(0:-92:1.25 and 0.5);
		\draw (1.25,0) arc(0:90:1.25 and 0.5);
		
		\draw (1.25,3.5) arc (0:-90:1.25 and 0.5);
		\draw [dashed] (1.25,3.5) arc (0:-90:1.25 and -0.5);
		\draw (1.25,3.5) -- (1.25,0);  
		
		\fill [opacity =0.6, color = blue] (1.25, 0) -- (1.25, 3.5) 
		arc(0:-90:1.25 and 0.5) -- (0, -0.5) node[midway, right, color =black]{} arc(270:360:1.25 and 0.5) ;
		\end{tikzpicture}

		
	\end{center}
-- грань тела, $2$-мерное многообразие.

\end{example}


\subsection{Касательное и нормальное пространства}

\begin{definition}
	Вектор $\vec{v} \in \mathbb{R}^n$ называется касательным к множеству 
	$M \subseteq \mathbb{R}^n$ в точке $x_0 \in M$, если существует кривая
	$\gamma : [0, \varepsilon] \to M$, такая что $\gamma(0) = x_0$ и $\gamma_t'(0) = \vec{v}$.
\end{definition}

Множество касательных векторов к $M$ в точке $x_0$ обозначается $T_{x_0}M$.

Колинеарный касательному вектору так же является касательным вектором, т.е. 
если $\vec{v} \in T_{x_0}M$, то $\lambda \vec{v} \in T_{x_0}M$.

\begin{theorem}[О множестве касательных векторов]
	Если $M \subseteq \mathbb{R}^n$ - $C^1$-гладкое $k$-мерное многообразие,
	$x_0 \in M$, тогда:
	
	\begin{enumerate}
		\item 
		Если $x_0 \in M \setminus \partial M$, то $T_{x_0}M \simeq 
		\mathbb{R}^k$.
		
		\item 
		Если $x_0 \in \partial M$, то $T_{x_0}M \simeq 
		\mathbb{R}_+^k$.
	\end{enumerate}
\end{theorem}

\begin{proof}
	По определению $k$-мерного многообразия, для каждого $x_0 \in M$ существует $U$ - окрестность $x_0$ и существует $ C^r$-диффеоморфизм $\Phi : U \to \Phi (U)$, такой что $\Phi(x_0)=0$ и $\Phi(U \cap M) = V \times \{0\}^{n-k}$, где $V$ - окрестность нуля в $R^n$.
	
	Под действием $\Phi$ кривая перейдет в кривую.
	
	$\gamma : [0, \varepsilon] \to M \implies \Gamma(t) = 
	\Phi(\gamma(t)) = (x_1(t), \ldots, x_k(t), 0, \ldots, 0) 
	\implies \Gamma'(t) = =(x'_1(t), \ldots, x'_k(t), 0, \ldots, 0)$.
	
	$\Gamma'(t) = \Phi(\gamma(t)) = D \Phi_{\gamma(t)} \langle\gamma'(t)\rangle$ -
	линейное отображение, такое, что $\det D \Phi \not = 0$.
	
	$T_{x_0}M = D \Phi^{-1} (\mathbb{R}^k \times \{0\}^{n-k}) \implies T_{x_0}M$ - $k$-мерная плоскость (т.к. дифференциал переводит плоскость в плоскость). 
\end{proof}

\begin{definition}
	Нормальное пространство $N_{x_0}M$ к дифференцируемому многообразию в точке $x_0$ -- это ортогональное дополнение к касательному пространству $T_{x_0}M$.
\end{definition}

\begin{lemma}
	Пусть $M \subset \mathbb{R}^n$ и $x_0 \in M$, тогда $\dim M = k$, 
	$\dim T_{x_0}M = k$, $\dim N_{x_0}M = n - k$.
\end{lemma}

\begin{theorem}[О базисе касательного пространства]
	Пусть $U \subset \mathbb{R}^k$ - открытое множество, $f : U \to \mathbb{R}^k$, $f \in C^r$. Тогда, если $M = f(U)$ - многообразие и $\rank Df = k$, то $\{\frac{\partial f}{\partial x_1} (t_0), \ldots, \frac{\partial f}{\partial x_k} (t_0) \}$ - базис в $T_{f(t_0)}M$.
\end{theorem}

\begin{proof}
	Пусть $t_0 \in U$, определим $\Gamma_j = t_0 + t \cdot \vec{e_j}$, тогда 
	$\gamma_j(t)=f(\Gamma_j(t))$ - кривая на многообразии.
	
	Найдем её касательный вектор в точке $t_0$: $\gamma_j'(t_0) = 
	\frac{d}{dt} f(t_0 + t \cdot \vec{e_j}) = \frac{\partial f}{\partial t_j}(t_0) \in T_{f(t_0)}M$.
	
	Набор векторов $\left\{\frac{\partial f}{\partial t_i}\right\}_{i=1}^{k}$ является линейно независимым и их количество равно размерности касательного пространства.
\end{proof}

\begin{theorem}
	Пусть многообразие $M$ задано системой уравнений:
	\begin{equation*}
	\begin{cases}
	f_1(x_1, \ldots, x_n) = a_1
	\\
	\ldots
	\\
	f_k(x_1, \ldots, x_n) = a_k
	\end{cases}
	\end{equation*}
	и $f_1, \ldots, f_k : U \to \mathbb{R}$, $f_i \in C^r \; \forall \, i \leq k$ , $U \subseteq \mathbb{R}^n$ -- открытое множество, а так же $\rank (\frac{\partial f_i}{\partial x_j})=k$, тогда система уравнений
	\begin{equation*}
	\begin{cases}
	df_1(x_0)\langle\vec{v}\rangle = 0
	\\
	\ldots
	\\
	df_k(x_0)\langle\vec{v}\rangle = 0
	\end{cases}
	\end{equation*}
	задает $T_{f(t_0)}M$, а $\{\nabla f_1(x_0), \ldots, \nabla f_k(x_0)\}$ базис в $N_{x_0}M$.
\end{theorem}

\begin{proof}
	Пусть $x_0 \in M$. Возьмем вектор $\vec{v} \in T_{f(t_0)}M$, тогда по определению существует кривая $\gamma : [0, \varepsilon] \to M$, такая что $\gamma(0) = x_0$ и $\gamma_t'(0) = \vec{v}$.
	
	$f(\gamma(t)) = 0 \implies 0=f(\gamma(t))'=df_{f(t)}\langle\gamma'(t)\rangle$.
	
	Подставим $t=0 \implies 0 = df_{x_0}\langle\vec{v}\rangle$. Из этого равенства следует, что: 
	
	\begin{equation*}
	\begin{cases}
	df_1(x_0)\langle\vec{v}\rangle = 0
	\\
	\ldots
	\\
	df_k(x_0)\langle\vec{v}\rangle = 0
	\end{cases}
	\end{equation*}
	
	Заметим, что $0 = df_{x_0}\langle\vec{v}\rangle = \nabla f_j(x_0) \cdot \vec{v}$, из чего получаем, что 
	\\
	$\{\nabla f_1(x_0), \ldots, \nabla f_k(x_0)\}$ -- базис в $N_{x_0}M$, т.к. $T_{f(t_0)}M \perp N_{x_0}M$.
\end{proof}

\subsection{Задача на условный экстремум}

\begin{definition}
	Пусть $M, N$ -- дифференцируемые многообразия, тогда $f : M \to N$ дифференцируема в точке $x_0$, если существует линейное отображение $L : T_{x_0}M \to T_{f(t_0)}N$, такое что для каждой кривой $\gamma : [0, \varepsilon] \to M$, такой что $\gamma \in C^1$, $\gamma(0)=x_0$, $\gamma'(0)=\vec{v} \in T_{x_0}M$, выполняется $f(\gamma(t)) = f(x_0) + tL(\vec{v}) + o(t)$.
\end{definition}

\begin{theorem}[Необходимое условие экстремума]
	Пусть $f: M \to \mathbb{R}$ - дифференцируема и $x_0$ - её экстремум, тогда $df(x_0) = 0$.
\end{theorem}

\begin{proof}
	Пусть $\vec{v} \in T_{x_0}M \Longleftrightarrow \exists $ кривая $\gamma : [0, \varepsilon] \to M$, такая что $\gamma(0) = x_0$ и $\gamma_t'(0) = \vec{v}$.
	
	Для $f \mid_{\gamma} x_0$ - экструмум, следовательно $f(\gamma(t))'(0)=0=df_{x_0}\langle\vec{v}\rangle$.
\end{proof}

Пусть $f: U \to \mathbb{R}$, $U \subseteq \mathbb{R}^n$ -- открытое множество, $M \subseteq U$ - $k$-мерное многообразие. Ставится задача: найти экстремум на многообразии $f\mid_{M}$.

\begin{theorem}[Необходимое условие условного экстремума]
	Если $x_0 \in M$ -- точка экстремума $f$, то $df\mid_{T_{x_0}M} = 0 \Leftrightarrow \nabla f(x_0) \in N_{x_0}M$.
\end{theorem}

\begin{proof}
	Пусть $\vec{v} \in T_{x_0}M \Longleftrightarrow \exists $ кривая $\gamma : [0, \varepsilon] \to M$, такая что $\gamma(0) = x_0$ и $\gamma_t'(0) = \vec{v}$.
	
	Тогда $x_0$ - экстремум $f(\gamma(t))$ и $f(\gamma(0)) = 0$. Заметим, что $f(\gamma(t))= df_{\gamma(0)}\langle \gamma'(0)\rangle=df_{x_0} \langle \vec{v} \rangle$.
\end{proof}

\begin{theorem}[Метод множителей Лагранжа]
	Пусть $f, \varphi_1, \ldots, \varphi_k : U \to \mathbb{R}$, $U \subseteq \mathbb{R}^n$. Тогда, если $x_0$ - условный экстремум при условиях $\varphi_1(\bar{x}) = 0, \ldots, \varphi_k(\bar{x})  = 0$, то $dL(x_0) = 0$, где $L(\bar{x}, \lambda_1, \ldots, \lambda_k)=f(\bar{x}) - \lambda_1\varphi_1(\bar{x}) - \ldots - \lambda_k\varphi_k(\bar{x})$ -- функция Лагранжа.
\end{theorem}

\begin{proof}
	Рассмотрим $L(\bar{x}, \lambda_1,\ldots,\lambda_k)$. $\frac{\partial L}{\partial \lambda_j}(x) = - \varphi_j(x) = 0 \Rightarrow x$ -- решение системы уравнений. 
	
	Возьмем частную производную $L$ по $x_j$:
	\[ \frac{\partial L}{\partial x_j} (x, \lambda) = \frac{\partial f}{\partial x_j}(x) - 
	\lambda_1\frac{\partial \varphi_1}{\partial x_j}(x) - 
	\ldots - \lambda_k \frac{\partial \varphi_n}{\partial x_j}(x) = 0, \]
	отсюда $\nabla f(x) = \lambda_1\nabla\varphi_1(x) + \ldots +
	\lambda_k\nabla\varphi_k(x) $ -- градиенты $\nabla\varphi_i$ -- это нормали, следовательно, их линейная комбинация тоже нормаль. Таким образом, $\nabla f$ -- нормаль к поверхности и выполнено необходимое условие условного экстремума.
	\end{proof}


% лекция 26.02

\begin{lemma}[Правило дифференцирования вдоль кривой] 
	\label{CurveDiffRule}
	\
	
	Пусть $f: U \to \mathbb{E} \in C^2, U \subseteq \mathbb{R}^n$ -- открытое множество 
	и $\gamma:[a, b] \to U \in C^2$ -- кривая. Тогда
	\begin{enumerate}
		\item 
		$(f\circ\gamma)'(t) = df_{\gamma(t)} \langle \gamma'(t)\rangle,$
		\item 
		$(f\circ\gamma)''(t) = d^2f_{\gamma(t)} \langle \gamma'(t), 
		\gamma'(t)\rangle + df_{\gamma(t)} \langle \gamma''(t) \rangle.$
	\end{enumerate}
\end{lemma}

	Есть два способа доказательства, здесь будет приведен самый оптимальный. Другой способ вы сможете найти в своих лекциях ;-)
\begin{proof}
	$(f \circ \gamma)''(t) = \left(df_{\gamma(t)}\langle \gamma'(t) \rangle\right)'_t $ -- дифференцирование сложной функции.
	
	Напомним, что $d_x\left(df_x\langle \vec{v}\rangle\right) = d^2f(x)\langle \vec{v} \rangle$ и
	$d_{\vec{v}} \left(df_x \langle \vec{v} \rangle\right) = df_x.$
	
	Отсюда имеем $\left( df_{\gamma(t)}\langle \gamma'(t)\rangle \right)'_t = 
	d^2f_{\gamma(t)} \langle \gamma'(t), \gamma'(t) \rangle + df_{\gamma(t)} \langle \gamma''(t) \rangle.$
	
\end{proof}

Теперь мы готовы сформулировать достаточное условие экстремума.

Перед доказательством заметим, что если $\varphi_1,\ldots, \varphi_n$ -- уравнения связи, то $x \in M$ (лежит в многообразии) тогда и только тогда, когда 
\begin{equation*}
\begin{cases}
\varphi_1(x) = 0,
\\
\ldots
\\
\varphi_n(x) = 0.
\end{cases}
\end{equation*}


\begin{theorem}[Достаточное условие экстремума в методе множителей Лагранжа]
	Пусть $U \subset \mathbb{R}^n$, $f, \varphi_1,\ldots, \varphi_k: U \to \mathbb{R} \in C^2$, 
	$\rank \left(\frac{\partial \varphi_i}{\partial x_j}\right) = k$ -- всюду, то функция Лагранжа 
	$L(x, \lambda_1,\ldots, \lambda_k) = f(x) - \lambda_1\varphi_1(x) - \ldots - \lambda_k \varphi_k(x) $
	если $dL(x_0,\lambda_0) = 0$, то при $d_x^2L(x_0, \lambda_0) > 0$ -- достигает минимума, при
	$d_x^2 L(x_0, \lambda_0) < 0 $ -- максимума.
\end{theorem}

\begin{proof}
	Пусть $dL(x_0,\lambda_0) = 0, d_x^2 L(x_0, \lambda_0)\mid_{T_{x}M\times T_{x}M} > 0$. 
	
	$ dL(x_0,\lambda_0) = 0 \iff \frac{\partial L}{\partial x} = 
	\frac{\partial f}{\partial x} - \sum_{j = 1}^{k} \lambda_j\frac{\partial \varphi}{\partial x_j} = 0
	\iff \nabla f(x_0) = \sum_{j = 1}^{k} \lambda_j\nabla\varphi_j.$ 
	Кроме того, $\frac{\partial L}{\partial x} = 0 = - \varphi_j \iff x_0 \in M$.
	Следовательно,  
	$\nabla f(x) \in N_{x_0}M \iff df_{x_0} \mid_{T_{x_0}M} = 0$.
	
	Возьмем произвольную $\gamma:(-\varepsilon, \varepsilon) \to M \in C^2, \gamma(0) = x_0$. 
	Посмотрим, как ведет себя функция $f$ на кривой $\gamma$, т.е. ограничение функции на эту кривую.
	
	Рассмотрим $f \circ \gamma : (-\varepsilon, \varepsilon) \to \mathbb{R}$. 
	По лемме \ref{CurveDiffRule} $(f\circ\gamma)'(t) = df_{\gamma(t)}\langle \gamma'(t)\rangle.$
	
	$(f \circ \gamma)'(0) = df_{x_0}\langle \gamma'(0)\rangle = 0$. 
	
	Поскольку образ кривой лежит в многообразии
	\begin{equation*}
	\gamma(t) = x \in M \iff
		\begin{cases}
		\varphi_1(\gamma(t)) \equiv 0, \\
		\ldots \\
		\varphi_k(\gamma(t)) \equiv 0.
		\end{cases}
	\end{equation*} 
	Подставим $\gamma(t) = x$ в $L(x, \lambda)$, получим
	\[ L(\gamma(t), \lambda) = f(\gamma(t)) - \lambda_1\varphi_1(\gamma(t)) - \ldots -
	\lambda_k\varphi_k(\gamma(t)) = f(\gamma(t)) \]
	далее,
	\[ (f \circ \gamma)''(t) = (L(\gamma(t), \lambda))'_{tt} =
	d_x^2 L\langle \gamma'(t), \gamma'(t) \rangle + d_x L\langle \gamma''(t) \rangle. \]
	подставим ноль
	\[ (f \circ \gamma)''(0) = d_x^2 L\langle \gamma'(0), \gamma'(0) \rangle + 0 > 0. \]
	Имеем, что $f \circ \gamma: (-\varepsilon, \varepsilon) \to \mathbb{R},
	 (f \circ \gamma)'(0) = 0, (f \circ \gamma)''(0) > 0$. 
	 Следовательно, $t = 0$ является точкой минимума для $f \circ \gamma$.
	 
	 Таким образом, $x_0$ -- точка минимума для любой кривой $\gamma \subseteq M$,
	 проходящей через $x_0$. Следовательно, $x_0$ -- точка минимума для $f\mid_{M}.$
\end{proof}

\begin{example}
	Найти экстремумы функции
	$f(x, y, z) = x^2 - 2x + y^2 - z^2$ на $x^2 + y^2 \leq 4, 0 \leq z \leq 1.$
\end{example}




\end{document}

